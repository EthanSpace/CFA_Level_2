\documentclass[cn,11pt,chinese]{elegantbook}

\title{CFA二级 学习笔记}
\subtitle{CFA Level 2 Learning Notes}

\author{Ethan Wang}
\institute{纽约大学}
\date{\today}
\version{0}
\bioinfo{鸣谢}{特别感谢本笔记模板制作者:Elegant\LaTeX{} Program }
\extrainfo{Vini. Vidi. Vici. --- Julius Caesar}

\logo{cover.png}
\cover{front.jpg}

% 本文档命令
\usepackage{array}
\newcommand{\ccr}[1]{\makecell{{\color{#1}\rule{1cm}{1cm}}}}
% 修改目录深度
\setcounter{tocdepth}{2}

\begin{document}

\maketitle
\frontmatter

\tableofcontents

\mainmatter 
\chapter{Introduction to Linear Regression}

\section{Linear Regression Introduction}
    一元回归模型应该看起来长这样:
    \begin{equation}
        \hat Y_i = \hat b_0 + \hat b_1 X_i + \epsilon_i, i = 1,\dots, n
    \end{equation}
    其中,\\
    \indent \(\hat Y_i\) (Predicted Value) 是对第\(i\)个因变量 (dependent variable)的估计\\
    \indent \indent \(\hat b_1\)的置信区间是
        \[\hat Y \pm (t_c \times s_f)\]
    \indent \indent \indent \(t_c\)是 two-tailed t-value检验值,自由度 (degree of freedom)是\(n - 2\)\\ 
    \indent \indent \indent \(s_f = \text{SEE}^2 [1 + \frac{1}{n} + \frac{(X - \overline{X})^2}{(n - 1)s_x^2}]\)是 standard error of the forecast,一般题目中会给\\ 
    \indent \indent \indent \indent \(s_x^2\)是 variance of the independent variable\\
    \indent \(\hat X_i\) 是对第\(i\)个自变量(independent variable)的估计\\
    \indent \(\hat b_1 = \text{COV}_{XY} / \delta_X^2\)是模型的坡度,slope coefficient.\\
    \indent \indent \(\hat b_1\)的置信区间是
        \[ \hat b_1 \pm (t_c \times s_{\hat b_1}) \] 
    \indent \indent \indent \(t_c\)是 two-tailed t-value检验值,自由度 (degree of freedom)是\(n - 2\)\\ 
    \indent \indent \indent \indent 所以检验\(\hat b_1\) 用\(t_{b_1} = \frac{\hat b_1 - b_1}{s_{\hat b_1}}\),并且拒绝\(H_0\)如果 \(t > |t_{critical}|\)\\ 
    \indent \indent \indent \(s_{\hat b_1}\)是 standard error of regression coefficient\\
    \indent \(\hat b_0 = \overline{Y} - \hat b_1 \overline{X}\)是模型的交点,intercept term.

\section{ANOVA Table}
    首先,我们先看看ANOVA Table是什么样子的,再解释里面的各项是什么意思
    \begin{table}[htbp]
        \centering
        \caption{ANOVA Table}
          \begin{tabular}{llll}
          \toprule
          Source of Variation & DoF (k) & Sum of Squares & Mean Sum of Squares \\
          \midrule
          Regression (explained) & 1 & RSS & MSR = \(RSS / k = RSS\)\\
          Error (unexplained) & n - 2 & SSE & MSE = \(\frac{SSE}{n - 2}\)\\
          \midrule
          Total & n - 1 & SST &   \\
          \bottomrule
          \end{tabular}%
        \label{tab:theorem-class}%
    \end{table}%

    \begin{enumerate}
        \item DoF (Degree of freedom) 自由度\\
                回归模型的自由度是自变量的个数,也就是\(k\);误差的自由度是(观测样本的个数) - (自变量个数) - 1,也就是 \(n - k - 1\)
        \item RSS (Regression sum of squares) 回归平方和\\
                回归模型能够解释的因变量变化量 
                \[RSS = \sum_{i = 1}^n (\hat Y_i - \overline{Y})^2\]
        \item SSE (Sum of squared errors) 残差平方和\\
                回归模型不能够解释的因变量变化量 
                \[SSE = \sum_{i = 1}^n (\hat Y_i - Y_i)^2\]
        \item SST (Total Sum of squares) 总离差平方和\\
                因变量的总变化量
                \[SST = \sum_{i = 1}^n (\overline{Y_i} - Y_i)^2\]
        \item MSR (Mean regression sum of squares) 平均回归平方和
        \item MSE (Mean squared error) 平均残差平方和
    \end{enumerate}
    其中,SS 代表 sum of squares。MS 代表 Mean sqaured,也就是对应的SS除以对应的自由度。

\section{Calculating \(R^2\) and SEE}
    \(R^2\) 是 coefficient of determination,表明多少百分比的数据可以被回归模型解释。这个值越大越好。\[R^2 = \frac{SST - SSE}{SST} = \frac{RSS}{SST}\]

    \(SEE\) 是standard deviation of the regression error terms,通过MSE开方得到.
    \[SEE = \sqrt{MSE} = \sqrt{\frac{SSE}{n - 2}}\]

\section{The F-Statistics}
    F值表明了该模型中的全部或一部分自变量是否适合用来估计母体。In multiple regression, the F-statistic is used to test whether at least one independent variable in a set of independent variables explains a significant portion of the variation of the dependent variable.
    \[F = \frac{MSR}{MSE} = \frac{RSS / k}{SSE / (n - k - 1)}\]

    F critical value 是通过分子和分母的自由度,还有significance level决定的。拒绝\(H_0\)的条件是\(F > F_c\)。

\chapter{Multiple Regression}
\section{Indroduction}
    多元回归一般的形式是 \[\hat Y_i = \hat b_0 + \hat b_1 X_{1i} + \hat b_2 X_{2i} + \hat b_k X_{ki}\]

\chapter{The Term Structure and Interest Rate Dynamics}
\section{Spot and Forward Rates}
    首先,我们先定义一下spot rate 和 forward rate 分别是什么。\\
    Spot rate 是未来单笔收入的年化市场利率。也可以理解成为零息债券的收益率。\\
    Forward rate 是对于某贷款事先定好的利率。

    \subsection{Spot Rates}
        我们定义,面值为1块钱的零息债券的现价为贴现因子(discount factor),记为\(P_T\)。同时,这笔投资的产出(yield to maturity)记为\(S_T\)
            \[P_T = \frac{1}{(1 + S_T)^T}\]
    
    \subsection{Forward Rates}
        


\end{document}
