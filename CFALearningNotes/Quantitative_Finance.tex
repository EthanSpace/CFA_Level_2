\documentclass[cn,11pt,chinese]{elegantbook}

\title{CFA二级 学习笔记}
\subtitle{CFA Level 2 Learning Notes}

\author{Ethan Wang}
\institute{纽约大学}
\date{\today}
\version{0}
\bioinfo{鸣谢}{特别感谢本笔记模板制作者:Elegant\LaTeX{} Program }
\extrainfo{Vini. Vidi. Vici. --- Julius Caesar}

\logo{cover.png}
\cover{front.jpg}

% 本文档命令
\usepackage{array}
\newcommand{\ccr}[1]{\makecell{{\color{#1}\rule{1cm}{1cm}}}}
% 修改目录深度
\setcounter{tocdepth}{2}

\begin{document}

\maketitle
\frontmatter

\tableofcontents

\mainmatter 
\chapter{INTRODUCTION TO LINEAR REGRESSION}

\section{Linear Regression Introduction}
    一元回归模型应该看起来长这样:
    \begin{equation}
        \hat Y_i = \hat b_0 + \hat b_1 X_i + \epsilon_i, i = 1,\dots, n
    \end{equation}
    其中,\\
    \indent \(\hat Y_i\) (Predicted Value) 是对第\(i\)个因变量 (dependent variable)的估计\\
    \indent \indent \(\hat b_1\)的置信区间是
        \[\hat Y \pm (t_c \times s_f)\]
    \indent \indent \indent \(t_c\)是 two-tailed t-value检验值,自由度 (degree of freedom)是\(n - 2\)\\ 
    \indent \indent \indent \(s_f = \text{SEE}^2 [1 + \frac{1}{n} + \frac{(X - \overline{X})^2}{(n - 1)s_x^2}]\)是 standard error of the forecast,一般题目中会给\\ 
    \indent \indent \indent \indent \(s_x^2\)是 variance of the independent variable\\
    \indent \(\hat X_i\) 是对第\(i\)个自变量(independent variable)的估计\\
    \indent \(\hat b_1 = \text{COV}_{XY} / \delta_X^2\)是模型的坡度,slope coefficient.\\
    \indent \indent \(\hat b_1\)的置信区间是
        \[ \hat b_1 \pm (t_c \times s_{\hat b_1}) \] 
    \indent \indent \indent \(t_c\)是 two-tailed t-value检验值,自由度 (degree of freedom)是\(n - 2\)\\ 
    \indent \indent \indent \indent 所以检验\(\hat b_1\) 用\(t_{b_1} = \frac{\hat b_1 - b_1}{s_{\hat b_1}}\),并且拒绝\(H_0\)如果 \(t > |t_{critical}|\)\\ 
    \indent \indent \indent \(s_{\hat b_1}\)是 standard error of regression coefficient\\
    \indent \(\hat b_0 = \overline{Y} - \hat b_1 \overline{X}\)是模型的交点,intercept term.

\section{ANOVA Table}
    首先,我们先看看ANOVA Table是什么样子的,再解释里面的各项是什么意思
    \begin{table}[htbp]
        \centering
        \caption{ANOVA Table}
          \begin{tabular}{llll}
          \toprule
          Source of Variation & DoF (k) & Sum of Squares & Mean Sum of Squares \\
          \midrule
          Regression (explained) & 1 & RSS & MSR = \(RSS / k = RSS\)\\
          Error (unexplained) & n - 2 & SSE & MSE = \(\frac{SSE}{n - 2}\)\\
          \midrule
          Total & n - 1 & SST &   \\
          \bottomrule
          \end{tabular}%
        \label{tab:theorem-class}%
    \end{table}%
    \(k\) is the number 




\end{document}